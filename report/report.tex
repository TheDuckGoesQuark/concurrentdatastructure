%%%%%%%%%%%%%%%%%%%%%%%%%%%%%%%%%%%%%%%%%
% University Assignment Title Page 
% LaTeX Template
% Version 1.0 (27/12/12)
%
% This template has been downloaded from:
% http://www.LaTeXTemplates.com
%
% Original author:
% WikiBooks (http://en.wikibooks.org/wiki/LaTeX/Title_Creation)
%
% License:
% CC BY-NC-SA 3.0 (http://creativecommons.org/licenses/by-nc-sa/3.0/)
%
%%%%%%%%%%%%%%%%%%%%%%%%%%%%%%%%%%%%%%%%%
%\title{Title page with logo}
%----------------------------------------------------------------------------------------
%	PACKAGES AND OTHER DOCUMENT CONFIGURATIONS
%----------------------------------------------------------------------------------------

\documentclass[12pt]{article}
\usepackage[english]{babel}
\usepackage[utf8x]{inputenc}
\usepackage{natbib}
\usepackage{amsmath}
\usepackage[colorinlistoftodos]{todonotes}
\usepackage{listings}
\usepackage{color}
\usepackage[explicit]{titlesec}
\usepackage{url}
\usepackage{subfig}
\usepackage{graphicx}
\usepackage{grffile}
\usepackage{mwe}

\titleformat{\section}{\normalfont\Large\bfseries}{Experiment \thesection}{1em}{}

\definecolor{dkgreen}{rgb}{0,0.6,0}
\definecolor{gray}{rgb}{0.5,0.5,0.5}
\definecolor{mauve}{rgb}{0.58,0,0.82}

\begin{document}

\begin{titlepage}

\newcommand{\HRule}{\rule{\linewidth}{0.5mm}} % Defines a new command for the horizontal lines, change thickness here

\center % Center everything on the page
 
%----------------------------------------------------------------------------------------
%	HEADING SECTIONS
%----------------------------------------------------------------------------------------

\textsc{\LARGE University of St Andrews}\\[1.5cm] % Name of your university/college
\textsc{\Large CS4204 Coursework 1}\\[0.5cm] % Major heading such as course name
\textsc{\large }\\[0.5cm] % Minor heading such as course title

%----------------------------------------------------------------------------------------
%	TITLE SECTION
%----------------------------------------------------------------------------------------

\HRule \\[0.4cm]
{ \huge \bfseries Concurrent Data Structure}\\[0.4cm] % Title of your document
\HRule \\[1.5cm]
 
%----------------------------------------------------------------------------------------
%	AUTHOR SECTION
%----------------------------------------------------------------------------------------


\Large \emph{Author:}\\
 \textsc{150008022}\\[3cm] % Your name

%----------------------------------------------------------------------------------------
%	DATE SECTION
%----------------------------------------------------------------------------------------

{\large \today}\\[2cm] % Date, change the \today to a set date if you want to be precise

%----------------------------------------------------------------------------------------
%	LOGO SECTION
%---------------------------------------------------------------------------------------

\includegraphics[width = 3.1cm]{images/standrewslogo.png}
 
%----------------------------------------------------------------------------------------

\vfill % Fill the rest of the page with whitespace

\end{titlepage}

\part*{Goal}

The goal of this practical was to implement and compare a locking and lock free version of a concurrent account data structure.

\part{Language and Structure}

To allow for thorough analysis, the C language was chosen for this practical. This would allow both data structure implementations to use fairly low level operations whilst maintaining a reasonable level of readability and ease of development. Since the gcc compiler can also provide the resulting assembly code, this could also be analysed when comparing lock free and locking implementations. Comparing the effects of compiler optimisation on the locking code could also be considered. 

The interface for the account is provided in \emph{account.h}. Since C doesn't support classes, the methods described in the practical specification were altered to also take an account pointer as an argument. A create and destroy method were also provided to allocate and free any memory used by the account struct.

\part{Locking}

The locking account struct involved two fields, the account balance and the account lock. This lock would have to be obtained in order for a method to make changes to the balance. The account creation method would initialise the mutex with the default attributes (PTHREAD_PRIO_INHERIT and PTHREAD_RECURSVIE_DISABLE). 




\part{Single Client Game}

\begin{figure}[!ht]
    \centering
    \begin{minipage}{0.45\textwidth}
        \centering
        \includegraphics[width=0.9\textwidth]{images/part1pretest} % first figure itself
        \caption{Before Press}
        \label{fig:beforepress}
    \end{minipage}\hfill
    \begin{minipage}{0.45\textwidth}
        \centering
        \includegraphics[width=0.9\textwidth]{images/part1posttest} % second figure itself
        \caption{After Press}
        \label{fig:afterpress}
    \end{minipage}
\end{figure}

\part*{Conclusion}

\bibliographystyle{unsrt}
\bibliography{mybib}

\end{document}
